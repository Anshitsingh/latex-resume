%%%%%%%%%%%%%%%%%%%%%%%%%%%%%%%%%%%%%%%%%%
%% Simple LaTeX Resume by Biswajit Paria %
%%%%%%%%%%%%%%%%%%%%%%%%%%%%%%%%%%%%%%%%%%

%%%%%%%%%%%%%%%%%%%%%%%%%%%%% Preamble %%%%%%%%%%%%%%%%%%%%%%%%%%%%%%%%%%%%%%%%%

\documentclass[10pt]{article}

\usepackage{graphicx}
\usepackage{amsmath}
\usepackage{amssymb}
\usepackage{algorithm, algorithmicx}
\usepackage{algpseudocode}
\usepackage{xcolor}
\usepackage{subcaption}
\usepackage{booktabs}
\usepackage{url}
\usepackage{parskip}
\usepackage[a4paper, total={7in, 10in}, top=1in]{geometry}

\pagenumbering{gobble}

\setlength{\parindent}{0pt}
\setlength{\itemsep}{0pt}
  \setlength{\parskip}{0pt}
  \setlength{\parsep}{0pt}
\renewcommand{\baselinestretch}{1}

\newcommand{\heading}[1]{
 {\large \textbf{#1}}
  \vspace{0.4em}
  \hrule
  \vspace{0.4em}
}

\newcommand{\EntryGap}{\vspace{0.4	cm}}
\newcommand{\SmallEntryGap}{\vspace{0.2cm}}
\newcommand{\mdot}{$\ \ \cdot\ \ $}

\newcommand{\indentedpar}[1]{
  \hangindent=1cm \hangafter=0 #1
}

\newenvironment{compactitemize}
{ \begin{itemize}
  \setlength{\itemsep}{0pt}
  \setlength{\parskip}{0pt}
  \setlength{\parsep}{0pt}}
{ \end{itemize}}


%%%%%%%%%%%%%%%%%%%%%%%%%%%%% Document %%%%%%%%%%%%%%%%%%%%%%%%%%%%%%%%%%%%%%%%%

\begin{document}



%%%%%%%%%%%%%%%%%%%%%%%%%%%%% Title %%%%%%%%%%%%%%%%%%%%%%%%%%%%%%%%%%%%%%%%%%%%

{\Large \textbf{Biswajit Paria}} \hfill biswajitsc@iitkgp.ac.in\\
5th year undergraduate (dual degree) \hfill biswajitsc@gmail.com\\
Dept. of Computer Science and Engineering \hfill \url{https://biswajitsc.github.io}\\
Indian Institute of Technology Kharagpur \hfill  +91-8348949676  
\EntryGap


%%%%%%%%%%%%%%%%%%%%%%%%%%%%% EDUCATION  %%%%%%%%%%%%%%%%%%%%%%%%%%%%%%%%%%%%%%%

\heading{Education}

\textbf{Indian Institute of Technology Kharagpur, India} \hfill Jul 2012 - Apr 2017\\
BTech MTech (dual degree) in Computer Science and Engineering\\
GPA 9.77/10.0, \textbf{ranked 1} among final year dual degree students.

\SmallEntryGap
\textbf{Kendriya Vidyalaya IIT Kharagpur} \hfill Jul 2010 - Apr 2012\\
All India Senior School Certificate Examination (AISSCE), CBSE board\\
Percentage score 92.4\%

\SmallEntryGap
\textbf{Kendriya Vidyalaya IIT Kharagpur} \hfill Apr 2005 - Apr 2010\\
All India Secondary School Examination (AISSE), CBSE board\\
GPA 9.8/10.0

\EntryGap
\heading{Interests}

My broad area of interest is machine learning. I am currently working on deep learning,
particularly on generative models including GANs and Variational Autoencoders.
In the future I want to work in more mathematically involved areas including statistical
machine learning, optimization, algorithms, and learning theory.


%%%%%%%%%%%%%%%%%%%%%%%%%% Papers %%%%%%%%%%%%%%%%%%%%%%%%%%%%%%%%%%%%%%%%%%%%%%
\EntryGap

\heading{Papers}
  
  Avisek Lahiri, \textbf{Biswajit Paria}, and Prabir Kumar Biswas. 
  ``Forward Stagewise Additive Model for Collaborative Multiview Boosting.''
  \emph{IEEE Transactions in Neural Networks and Learning Systems}. 2016.
  (Accepted, in press)
  
  \SmallEntryGap
  Avisek Lahiri, \textbf{Biswajit Paria}, and Prabir Kumar Biswas.
  ``CoMA-Boost: Cooperative Multi Agent AdaBoost.''
  \emph{Indian Conference on Computer Vision, Graphics and Image Processing}. 2016.
  (Accepted, oral presentation)
  
  \SmallEntryGap
  \textbf{Biswajit Paria}, Anirban Santara, and Pabitra Mitra.
  ``Visualization Regularizers for Neural Network based Image Recognition.''
  \emph{arXiv preprint arXiv:1604.02646}. 2016. (Pre-print)
  
  \SmallEntryGap
  \textbf{Biswajit Paria}, Sanjoy Pratihar, and Partha Bhowmick.
  ``On Farey Table and its Compression for Space Optimization with Guaranteed Error Bounds.''
  2016. (Submitted. Please request to get a copy for personal viewing.)


%%%%%%%%%%%%%%%%%%%%%%%%%%%%%%%% INTERNSHIPS %%%%%%%%%%%%%%%%%%%%%%%%%%%%%%%%%%%
\EntryGap

\heading{Internships and Projects}

\textbf{Bachelor's Thesis, Indian Institute of Technology Kharagpur} \hfill 2015 - 2016\\
\textbf{Visualization Regularizers for Neural Network based Image Recognition}\\
Advisor: Prof. Pabitra Mitra

\SmallEntryGap

\indentedpar{
    Introduced a novel regularizer for Neural Networks
    based on the \emph{visualizations} of the hidden nodes. 
    We leveraged the closed algebraic form of the
    visualizations of the first layer nodes, and used it as a smoothness prior.
    We show that the regularizer is a special case of the 
    general Tikhonov regularization. Experimental results show that the 
    visualization regularizers are an improvement over standard L2 regularizers.
}

\SmallEntryGap

\textbf{Indian Institute of Technology Kharagpur, India} \hfill 2015 - 2016\\
\textbf{Forward Stagewise Additive Model for Collaborative Multiview Boosting.}\\
with Prof. P.K. Biswas

\SmallEntryGap

\indentedpar{
  On AdaBoost in a multiview setting.
  Introduced a notion of ‘difficulty’ of an example. Difficulty ranges 
  from -1 to 1 and is a linear function of the number of views the example
  was misclassified in. The weights are updated as a function of the 
  difficulty the example rather than on whether the example was misclassified
  or not. We show that this is an improvement over other multiview frameworks. Analogous
  to AdaBoost, we derived bounds on 
  the empirical risk and maximum margin.
}

\SmallEntryGap

\textbf{Indian Institute of Science, Bangalore, India} \hfill Summer 2016\\
\textbf{Natural Language Inference (NLI) using Deep Learning for NLP}\\
with Prof. Ambedkar Dukkipati

\SmallEntryGap

\indentedpar{
  Worked with the Stanford NLI dataset. Given two sentences, the task
  is to determine whether the sentences are related as an entailment, contradiction
  or neutral. We explored models using LSTMs, Attentions, and Dependency Parses,
  and finally came up with a hybrid model. This work
  has been submitted to a reputed conference.
}

\SmallEntryGap

\textbf{University of Southern California, Los Angeles, USA} \hfill Summer 2015\\
\textbf{Feature Learning in Clinical Time Series using Deep Learning}\\
with Prof. Yan Liu

\SmallEntryGap

\indentedpar{
  Worked with ICU time series data consisting and used deep learning with a laplacian 
  prior regularizer to predict health outcomes. We performed causality analysis on the 
  final layer nodes, analyzed the activations of the most \emph{causal} nodes 
  using decision trees, and extracted their maximally activating inputs. This analysis
  is a step towards understanding neural networks in the context of medical data.
}

\SmallEntryGap
\textbf{Indian Institute of Technology Kharagpur, India} \hfill Summer 2014\\
\textbf{On Farey Table and its Compression for Space Optimization with Guaranteed Error Bounds}\\
with Prof. Partha Bhowmick

\SmallEntryGap

\indentedpar{
  Studied the number theoretic properties of Farey Sequences and the
  Farey Table, and came up with a novel algorithm for a lossy compression of the 
  Farey Table. The Farey Table is an useful data-structure in digital geometry.
  It's quadratic size ($\Theta(n^2)$) prohibits its use for large dimensions. The
  compressed table has a size of ($O(n\log n)$), thus allowing table creation for
  larger $n$.
}

\SmallEntryGap

\textbf{Other Projects}

\SmallEntryGap

\indentedpar{
  I have worked on numerous other projects, including developmental projects. My GitHub
  profile (\url{https://github.com/biswajitsc}) lists most of the projects I have worked on.
}


%%%%%%%%%%%%%%%%%%%%%%%%%%%%%%%% ACADEMIC HONORS & AWARDS %%%%%%%%%%%%%%%%%%%%%%


\EntryGap
\heading{Academic Honors and Awards}
 
 \textbf{Goralal Syngal Scholarship} \hfill 2015 \& 2016\\
 for academic excellence at IIT Kharagpur.
 \SmallEntryGap
  
 \textbf{Viterbi-India Scholarship} \hfill 2015\\
  for funding a summer internship at USC. One of the 20 scholars in India.
  \SmallEntryGap
  
 \textbf{ACM ICPC 2015 and 2016 World Finalist} \hfill 2015\\
  Our team qualified for the International Collegiate Programming Competition (ICPC)\\
  twice, in 2015 and 2016. One of the 7 teams from India.
  \SmallEntryGap
  
 \textbf{JBNSTS third best project} \hfill 2014\\
  Awarded for our work on Counting Dyck Paths of Bounded Height.
  \SmallEntryGap
  
 \textbf{Jagadish Bose National Science Talent Search (JBNSTS) Scholar} \hfill 2013\\
  Awarded to 30 candidates in the state of West Bengal.
  \SmallEntryGap
  
 \textbf{Indian National Physics Olympiad (INPhO) Awardee} \hfill 2012\\
  Among top 30 candidates in India.\\
  Selected to attend the 
  International Physics Olympiad (IPhO) selection camp.
  \SmallEntryGap
  
 \textbf{Kishore Vaigyanik Protsahan Yojana (KVPY) Scholar} \hfill 2011\\
  by Dept. of Science and Technology, Govt. of India for exceptional aptitude in basic sciences.\\
  Stood 7th in India.
  \SmallEntryGap
  
 \textbf{Indian National Mathematical Olympiad (INMO) Awardee} \hfill 2010\\
  Among top 30 candidates in India.\\
  Selected to attend the International Mathematical Olympiad Training Camp (IMOTC).
  \SmallEntryGap
  
 \textbf{Australian Mathematics Competition (AMC) Gold Medallist}\hfill 2009\\
  Recieved a Gold Medal in the Intermediate Division. One of the 23 medallists in the world.



%%%%%%%%%%%%%%%%%%%%%%%%%%%%%%%%%%%%%% SKILLS %%%%%%%%%%%%%%%%%%%%%%%%%%%%%%%%%%

\EntryGap
\heading{Technical Skills}
\SmallEntryGap
\textbf{Proficient:} C, C++, Python, Java, Bash, Matlab, Tensorflow, Numpy\\
\textbf{Familiar:} Mathematica, HTML, Javascript, Caffe, Theano, Scikit-learn,
Keras, Lasagne, Nltk, Stanford Core-NLP



%%%%%%%%%%%%%%%%%%%%%%%%%% RELEVANT COURSES   %%%%%%%%%%%%%%%%%%%%%%%%%%%%%%%%%%


\EntryGap
\heading{Relevant Coursework}

\SmallEntryGap
\begin{tabular}{lll}
 Machine Learning & Advanced Machine Learning & Probability and Statistics\\
 Computational Statistics & Algorithms I and II & Selected Topics in Algorithms\\
 Artificial Intelligence & Matrix Algebra & Information Retrieval\\
 Speech and Natural Language Processing & Advanced Graph Theory
\end{tabular}

% \SmallEntryGap
% \textbf{Others}\\
% Algorithms-I \& II \mdot Discrete Mathematics \mdot Parallel and Distributed Algorithms 
% \mdot Selected Topics in Algorithms  \mdot Computational Statistics
% \mdot Advanced Graph Theory \mdot Database Management Systems \mdot Operating Systems
% \mdot Computer Networks \mdot Computer Organization and Architecture \mdot Theory of Computation
% \mdot Operations Research \mdot High Performance Computer Architecture \mdot Distributed Systems
% \mdot Cryptography

\end{document}
